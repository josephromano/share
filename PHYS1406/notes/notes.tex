%\documentclass[10pt]{article}
\documentclass[10pt,twoside]{article}
\usepackage{amssymb,amsmath,amsthm,longtable}
\usepackage{hyperref}
\usepackage{latexsym}
\usepackage{placeins}
\usepackage{graphicx}
\usepackage{caption}
\usepackage{textcomp}
\usepackage{enumerate}
\captionsetup{width=5.75in}
\usepackage[top=1in, bottom=1in, left=1in, right=1in]{geometry}
\usepackage{draftwatermark}
\SetWatermarkText{DRAFT}
\SetWatermarkColor[gray]{0.95}
\SetWatermarkScale{8}

\numberwithin{equation}{section}

% begin equation, itemize, etc.
\def\be{\begin{equation}}
\def\ee{\end{equation}}
\def\bi{\begin{itemize}}
\def\ei{\end{itemize}}
\def\ben{\begin{enumerate}}
\def\een{\end{enumerate}}
\def\i{\item{}}

\def\demo{\underline{Demonstration}:\ }
\def\defn{\underline{Definition}:\ }
\def\ex{\underline{Example}:\ }
\def\exer{\underline{Exercise}:\ }
\def\soln{\underline{Solution}:\ }
\def\ques{\underline{Question}:\ }
\def\ans{\underline{Answer}:\ }
 
%%%%%%%%%%%%%%%%%%%%%%%%%%%%%%%%%%%%%%%%%%%%%%%%%%%%%%%%%%%%%%
            
\begin{document}

\title{PHYS 1406: Physics of Sound \& Music\\
(Additional Lecture Notes)}
\author{J.D.\ Romano}
\date{Spring 2020}
\maketitle
 
\begin{abstract}
\noindent
{\bf Disclaimer:} 
These notes are meant to supplement the textbook
``Physics of Sound \& Music" by Prof.~Borst.
Please send corrections, comments, criticisms, suggestions to:
{\tt joseph.d.romano@ttu.edu}.

\end{abstract}

\cleardoublepage
\tableofcontents

\cleardoublepage
\part{Introduction}

\cleardoublepage
\input preliminaries

\cleardoublepage
\part{Physics of oscillations and waves}

\cleardoublepage
\input oscillations

\cleardoublepage
\input waves

\cleardoublepage
\input fourier

\cleardoublepage
\part{Production of sound}

\cleardoublepage
\input strings

\cleardoublepage
\input wind

\cleardoublepage
\input percussion

\cleardoublepage
\input piano

\cleardoublepage
\input voice

\cleardoublepage
\part{Perception of sound}

\cleardoublepage
\input hearing

\cleardoublepage
\input loudness

\cleardoublepage
\input pitch

\cleardoublepage
\part{Room acoustics, reproduction of sound}

\cleardoublepage
\input room-acoustics

\cleardoublepage
\input electrical-reproduction

\cleardoublepage
\part{Music theory}

\cleardoublepage
\input scales

\cleardoublepage
\input tuning

%%%%%%%%%%%%%%%%%%%%%%%%
\cleardoublepage
\begin{appendix}

\cleardoublepage
\input math-details

\cleardoublepage
\input routines

\cleardoublepage
\input hw1

\cleardoublepage
\input hw2

\cleardoublepage
\input hw3

\cleardoublepage
\input hw4

\cleardoublepage
\input hw5

\cleardoublepage
\input hw6

\cleardoublepage
\input hw7

\cleardoublepage
\input hw8

\cleardoublepage
\input hw9

\cleardoublepage
\input hw10

\cleardoublepage
\input hw11

\cleardoublepage
\input hw12

\cleardoublepage
\input hw13
\end{appendix}

\end{document}

