\section{Homework 12}

\noindent
Show all work.

\ben

\i List the main vocal organs.

\i Explain the Bernoulli effect.

\i Is it possible to produce sounds that do not require the
vocal folds to vibrate?
If so, give an example.

\i What are formants and hope do they shape the spectrum of the
radiated sound?

\i What distinguishes different vowel sounds from one another?

\i Describe a simple model of the human vocal tract that 
predicts formant peak frequencies at 500~Hz, 1500~Hz, 2500~Hz, etc.

\i What is a sound spectrogram? 
Why is it more useful than a single spectrum?

\i If I record myself whistling at 1000~Hz, what type of filter 
will reproduce the recorded sound: 
\ben
\i a low pass filter with cutoff frequency at 500~Hz.
\i a high pass filter with cutoff frequency at 1500~Hz.
\i a band pass filter with a low cutoff frequency at 500~Hz
and a high cutoff frequency at 1500~Hz.
\een

\i What is a singer's formant, and why is it useful for an
opera singer to be able to produce one?

\i How can a soprano tune the peak of her first formant region
to match the fundamental frequency of a high-pitched note?

\een

