\section{Homework 1}

\noindent
Show all work.

\ben
\i Write the following numbers in scientfic notation:
(a) 2~gigabytes,
(b) 5~msec.

\i Evaluate the following quantity both as simple fraction 
and as a decimal
number. 
$$
\left(\frac{3}{2}\right)^4\cdot \left(\frac{1}{2}\right)^2
$$
Approximately, what frequency interval does this
correspond to---an octave, a perfect fifth, a major third, 
a minor third, $\cdots$\ ?

\i What physical quantity corresponds to pitch?

\i One note is 10 octaves higher in pitch than another note.
What frequency ratio does this correspond to?

\i Sound in dry air at 25$^\circ{\rm C}$ travels at 346~{\rm m/s}.
(a) Convert this to ft/s.
(b) Determine how long it would take sound to make a 
roundtrip to a wall 50~ft away.

\i Light travels at approximately $3\times 10^8~{\rm m/s}$.
Given that the distance between the Earth and Moon is 
approximately $d=384400~{\rm km}$, determine how long it takes 
light to make a round trip from the Earth to moon.
(Hint: Don't forget to convert km to m.)

\i The musical note A$_2$ has a fundamental frequency of 110~Hz.
(a) Calculate the first eight harmonics of A$_2$.
(b) How many octaves above A$_2$ is the eigth harmonic?

\i Concert A$_4$ has a fundamental frequency of 440~Hz.
Calculate the period in 
(a) seconds, 
(b) milliseconds.

\i What's the main feature that distinguishes a musical note
from noise?

\i Why does the same musical note, for example C$_4$, sound 
differently when it is played on two different instruments---e.g., 
a flute and a clarinet?

\een

