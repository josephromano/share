\section{Homework 9}

\noindent
Show all work.

\ben

\i What two things (in addition to the speed of sound
in air) do the vibration frequencies of an air column 
in a tube depend on?

\i What type of tube produces only odd harmonics?

\i Is an open end of a tube a node for air-molecule motion
or pressure deviation?

\i What happens to a positive-pressure pulse when it
reflects off
(a) an open end of a tube, 
(b) a closed end of a tube.

\i Give an example of an instrument that uses 
(a) a flow-controlled excitation, 
(b) a pressure-controlled excitation.

\i A flute and a clarinet both have a length of 66~cm.
Calculate the fundamental frequency of 
(a) the flute, 
(b) the clarinet when all the tone holes are closed.

\i How big must a tone hole be in order for the
effective length of the tube to end at the tone
hole?

\i What's the purpose of a register key on a clarinet?
How does it work?

\i What does it mean for a brass player ``to play harmonics"?

\i Why doesn't a trumpet produce only odd harmonics.

\een

