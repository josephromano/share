\section{Homework 13}

\noindent
Show all work.

\ben

\i Calculate the sound power level $L_W$ of a 1-watt 
omni-directional loudspeaker.

\i Calculate the sound pressure levels $L_p$ of the 
loudspeaker from the previous problem at distances
of 1~m, 2~m, and 4~m from the loudspeaker.

\i You are standing 50~ft in front of a high cliff.
If you clap your hands, how long does it take for 
the reflected sound to reach your ears?
Will you hear an echo?

\i Explain the difference between direct, early 
reflected, and reverberant sound.

\i You need to convert a concert hall into a room
that will be used for large public speaking events.
What acoustical changes would you make?

\i Calculate the reverberation time at 500~Hz for a room 
having dimensions $8~{\rm m}\times 10~{\rm m}\times 3~{\rm m}$ (high),
concrete walls (0.06), plywood floor (0.17), and suspended
acoustical tile ceiling (0.83).
Also, assume that there are 50~wooden chairs (0.02~m${}^2$/chair)
and 30~students (0.39~m${}^2$/student).

\i Calculate the sound pressure level of the reverberant
sound at 500~Hz for a 1-watt omni-directional loudspeaker in a room that
has a surface area $S=1000~{\rm m}^2$ and a total absorption
$A=400~{\rm m}^2$ at 500~Hz.
(Hint: You will first need to calculate the average 
absorption $\bar a$ and the room constant $R$.)

\i What is the critical distance from the loudspeaker 
for the above problem?
 
\i Why do you sound better when you sing in the shower?

\i Explain in what sense `liveness' and `clarity' of sound
are competing acoustical properties.

\een

