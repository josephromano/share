\section{Mathematical details}

\bi

\i Logarithm identities:
%
\begin{align}
\log(ab)  &= \log a + \log b
\\
\log(a/b) &= \log a - \log b
\\
\log(a^b) &= b\log a
\\
a^{\log b} &= b^{\log a}
\end{align}
%
These are true for logarithms with respect to any base.

\i Trigonometric identities:
%
\begin{align}
\sin A\,\sin B &=\frac{1}{2}\left[\cos(A-B)-\cos(A+B)\right]
\\
\cos A\,\cos B &=\frac{1}{2}\left[\cos(A-B)+\cos(A+B)\right]
\\
\sin A\,\cos B &=\frac{1}{2}\left[\sin(A-B)+\sin(A+B)\right]
\end{align}
%

\i Adding two sinusoids (same frequency):

The sum of two sine functions having the same 
frequency (but possibly different amplitudes and
phases) yields another sinusoid with 
the {\em same} frequency:
%
\be
A_1\sin(2\pi ft+\phi_1) + A_2\sin(2\pi ft+\phi_2)
= A \sin(2\pi ft + \phi)
\ee
%
where
%
\be
A = \sqrt{A_1^2+A_2^2 + 2 A_1 A_2\cos(\phi_1-\phi_2)}\,,
\quad
\tan\phi = \frac{A_1\sin\phi_1+A_2\sin\phi_2}{A_1\cos\phi_1 + A_2\cos\phi_2}
\ee

\i Adding two sinusoids (different frequencies):

The sum of two sinusoids with the same amplitude
but {\em different} frequencies and phases is given by
%
\be
A\sin(2\pi f_1 t+\phi_1) + A\sin(2\pi f_2 t+\phi_2)
=2A\cos\left(\pi\Delta ft + \frac{(\phi_2-\phi_1)}{2}\right)
\sin\left(2\pi\bar ft + \frac{(\phi_1+\phi_2)}{2}\right)
\ee
%
where
%
\be
\bar f =\frac{1}{2}(f_1+f_2)\,,
\quad
\Delta f = f_2-f_1
\ee
%
If $\bar f$ is in the audio range (between about 20~Hz 
and 20~kHz) and $|\Delta f|\lesssim 15~{\rm Hz}$, we hear 
{\em beats} with frequency 
%
\be
f_{\rm beat} = |\Delta f| = |f_2-f_1|
\ee
%
The beats are amplitude modulations of a tone with average
frequency $\bar f$.

Note that the modulated signal is periodic only if 
$f_1/f_2 = m/n$ for some positive integers $m$ and $n$
({\em comensurate} frequencies).
For such a case, the period of the modulated signal 
is given by $T=mT_1=nT_2$, where $T_1$ and $T_2$ are the 
periods of the two component sinusoids---i.e.,
$T_1=1/f_1$ and $T_2=1/f_2$.

\i Amplitude modulation:

The product of two sine functions having different 
frequencies and phases is given by:
%
\begin{align}
\sin(2\pi f_1 t + \phi_1)\sin(2\pi f_2 t + \phi_2)
=
&\frac{1}{2}\big(
\cos\left[2\pi(f_2-f_1)t+(\phi_2-\phi_1)\right]
\nonumber\\
&\quad\quad\quad 
-\cos\left[2\pi(f_1+f_2)t+(\phi_1+\phi_2)\right]
\big)
\end{align}
%
Note the presence of {\em sidebands} at the sum and
difference frequencies $f_2-f_1$ and $f_2+f_1$.
This is a non-linear combination of sinusoids, as opposed
to a linear combination such as addition of two sinusoids.

\i Displacement of air molecules:

For a sinusoidal wave propagating in the $x$-direction, 
the displacement of an air molecule away from its equilibrium 
position at $x$ is given by:
%
\be
s(x,t) = s_m \,\sin \left(k x-\omega t\right)
\ee
%
where $k=2\pi/\lambda$ and $\omega=2\pi/T$.
(Note: $\omega/k=v$.)
It follows that
%
\begin{align}
\frac{\Delta s}{\Delta x} 
= k s_m\,\cos \left(k x-\omega t\right)\,,
\quad
u\equiv
\frac{\Delta s}{\Delta t} 
= -\omega s_m\,\cos \left(k x-\omega t\right)
\end{align}
%
Thus,
%
\be
\frac{\Delta s}{\Delta x}
=-\frac{k}{\omega}\frac{\Delta s}{\Delta t}
=-\frac{1}{v}\frac{\Delta s}{\Delta t}
=-\frac{u}{v}
\ee
%

\i Relationship between $\Delta V/V$ and $\Delta s/\Delta x$:
%
\be
V = A\Delta x\,,
\quad
\Delta V = A\Delta s
\quad\Rightarrow\quad
\frac{\Delta V}{V} = \frac{\Delta s}{\Delta x}
\ee

\i Relationship between pressure $p$ and displacement $s$:
%
\be
p = -B\frac{\Delta V}{V}\,,
\quad
\frac{\Delta V}{V} = \frac{\Delta s}{\Delta x}=-\frac{u}{v}\,,
\quad
v=\sqrt{\frac{B}{\rho}}
\quad\Rightarrow\quad
p = \rho v u = \rho v\frac{\Delta s}{\Delta t}
\ee
%
Thus, $p$ is proportional to $u$, 
which means that $p$ and 
$s$ are 90 degrees out of phase with one another.

\i Relationship between intensity $I$ and pressure $p$:
%
\be
I =({\rm energy\ density})\cdot v
=2\cdot \frac{\frac{1}{2}mu^2}{V}\cdot v
=\rho u^2 v 
= \frac{p^2}{\rho v}
\ee
%
Thus, $I$ is proportional to $p^2$.

\i Fourier series:

Any periodic function $y(t)$ with period $T$ can be
written as a sum of sines and cosines:
%
\be
y(t) = A_0 
+\left[a_1\sin(2\pi f_1 t) + b_1\cos(2\pi f_1 t)\right]
+\left[a_2\sin(2\pi f_2 t) + b_2\cos(2\pi f_2 t)\right]
+ \cdots
\ee
%
where $f_n=n f_1$ ($n=1,2\cdots$) 
are harmonic frequencies of the fundamental $f_1=1/T$.

Alternatively, one can write
%
\be
y(t) = A_0 
+ A_1\sin(2\pi f_1 t + \phi_1) 
+ A_2\sin(2\pi f_2 t + \phi_2) 
+ \cdots
\ee
%
where
$A_n$ and $\phi_n$ are the amplitudes and phases of the 
sinusoids.

In terms of $a_n$ and $b_n$,
%
\be
A_n= \sqrt{a_n^2+b_n^2}\,,\quad
\tan\phi_n = b_n/a_n
\ee

\i Plucked guitar string:

The vibrations of a plucked guitar string (length $L$)
can be written as a sum of standing wave vibrations
%
\be
y(x,t) = 
A_1\sin(2\pi x/\lambda_1)\cos(2\pi f_1 t) 
+A_2\sin(2\pi x/\lambda_2)\cos(2\pi f_2 t)
+\cdots
\ee
%
where $A_n$ ($n=1,2,\cdots)$
are the amplitudes of the standing wave vibrations, 
having wavelengths 
$\lambda_n = 2L/n$ and frequencies 
$f_n = v/\lambda_n=nv/2L = nf_1$ 
($v$ being the wave velocity on the string). 

Note that each term in the sum can be written as the
sum of a right-moving and left-moving wave:
%
\be
A_n\sin(2\pi x/\lambda_n)\cos(2\pi f_n t)
= \frac{1}{2}\left[
\sin(k_n(x-vt))+\sin(k_n(x+vt))
\right]
\ee
%
where $k_n= 2\pi/\lambda_n = n\pi/L$.

\ei
