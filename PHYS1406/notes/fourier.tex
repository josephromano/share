\section{Fourier analysis}
\label{s:fourier}

\bi

\i {\em Fourier's Theorem}:
Any complex periodic function can be
written as a sum of harmonics.

\i {\em Corollary to Fourier's Theorem}:
Any complex periodic vibration can be
written as a sum of standing wave vibrations.

\i Mathematically, Fourier's theorem can be expressed as
%
\be
y(t) = A_0
+ A_1\sin(2\pi f_1 t + \phi_1)
+ A_2\sin(2\pi f_2 t + \phi_2)
+ \cdots
\ee
%
where $y(t)$ is a periodic function with period $T$
and $f_n=n f_1$ ($n=1,2\cdots$) are harmonics of 
the fundamental frequency $f_1= 1/T$.
The term $A_0$ allows for a constant offset, and 
$A_n$ and $\phi_n$ are the amplitudes and phases of 
the component sinusoids.

\i NOTE: The above theorem also applies to periodic
functions of position $y(x)$, where $L$ is the 
period of the function.

\i Mathematically, the corollary to Fourier's theorem
can be expressed as
%
\be
y(x,t) =
A_1\sin(2\pi x/\lambda_1)\cos(2\pi f_1 t + \phi_1)
+A_2\sin(2\pi x/\lambda_2)\cos(2\pi f_2 t + \phi_2)
+\cdots
\ee
%
where $A_n$ and $\phi_n$ ($n=1,2,\cdots)$
are the amplitudes and phases of the standing 
wave vibrations, having wavelengths
$\lambda_n = 2L/n$ and frequencies
$f_n = v/\lambda_n=nv/2L$
($v$ being the wave velocity on the string).

\i Each term above corresponds to a standing
wave vibration pattern on the string, with a vibration
frequency that is a harmonic of the fundamental
frequency $f_1=v/2L$.


\i Decomposing a periodic function into its component
sinusoids is called {\em Fourier analysis}.
Constructing a periodic function by adding to 
together harmonics having different amplitudes and
phases is called {\em Fourier synthesis}.

\i If the function $y(t)$ is not periodic, one can
still decompose it into sinusoids, but now one has
contributions (in principle) from {\em all} frequencies,
and not just the (discrete) harmonics.

\i \demo The FFT Analyzer tool from the Faber 
Electroacoustics Toolbox performs Fourier analysis.

A pure tone (e.g., a tuning fork) consists 
of just a single frequency component.
A note produced by an instrument (e.g., a guitar or
recorder) will have a frequency spectrum dominated by
a few harmonics.

\i Examples of different waves and their Fourier
decompositions:

(i) square wave: only odd harmonics, with 
amplitudes 1, 1/3, 1/5, $\cdots$, and 
phases all 0.

(ii) triangle wave: only odd harmonics, with
amplitudes 1, 1/9, 1/25, $\cdots$, and 
phases alternating between 0 and 180$^\circ$ 
(so alternating $\pm$).

(iii) sawtooth wave: both odd and even harmonics, with 
amplitudes 1, 1/2, 1/3, $\cdots$, and 
phases all 0.

\i \demo Use the matlab routines
fouriersythesize.m, fouriersynthesizeScript.m to 
synthesize periodic waves given the amplitudes and
phases of the harmonics.

\i \demo Convert these to periodic functions to
audible sound using the matlab routines
fouriersythesizesound.m, fouriersynthesizesoundScript.m.

\i \demo Synthesize sound waves having the same
component amplitudes but different component phases.
Can you hear a difference?

Typically, phase has little effect on the timbre of a 
sound.
This is called {\em Ohm's law of hearing.}
The timbre is only noticeably different 
when the phases of two Fourier combinations 
are radically different from one another---e.g.,
both waves have amplitudes $1, 1/2, 1/3, \cdots$ 
but phases $0, 0, 0, \dots$ 
versus $0, 90^\circ, 0, 90^\circ, \cdots$ (so alternating
sines and cosines).
 
\i \demo Synthesize sound waves having a contributions
from the 2nd, 3rd, and higher harmonics, but no contribution
from the fundamental---e.g., 
contributions from 440~Hz, 660~Hz, 880~Hz, $\cdots$ 
but not from 220~Hz.
This is an example of a sound have a 
{\em missing fundamental}.
What pitch do you hear?
Do you hear the fundamental frequency even though it 
has zero amplitude?

NOTE: Most people hear a pitch corresponding to the 
fundamental frequency even though it is absent from 
the spectrum.
This is called a {\em virtual} or {\em subjective} pitch.
Although the ear doesn't physically respond to a 
frequency that's not present, cognitive processes in the 
brain infer its presence from the timing of electrical 
impulses triggered by the periodicity of the sound wave.
%[I'm not sure right now if this subjective pitch 
%corresponds to a {\em real} difference tone
%of $220~{\rm Hz}= (660-440)~{\rm Hz}$ produced
%by non-linearities in the ear (basically amplitude 
%modulation between the 660~Hz and 440~Hz components), 
%or is the result of cognitive processes in the brain.]

\ei
