\section{Homework 4}

\noindent
Show all work.

\ben

\i (a) 
What type of tube (open at both ends or
closed at one end) produces sound with only odd
harmonics?
(b) Give an example of a wind instrument that
produces only odd harmonics.

\i What is the difference between harmonics and
overtones?

\i Standing waves are set up in a column of air
in a tube that is closed at one end.
Is the closed end a node or anti-node for 
(a) air molecule displacement,
(b) pressure deviation from atmospheric pressure?

\i You clap your hands and hear an echo from a nearby
canyon wall 0.50~sec later.
How far away is the wall?
Assume $v=346~{\rm m/s}$ for the speed of sound in air.
(Hint: Don't forget that sound must travel to the 
canyon wall and back to be heard as an echo.)

\i Does sound tend to carry when there is a temperature
inversion (i.e., when the temperature increases with
height above the ground) or when the temperature 
decreases with increasing height above the ground?
Explain why it carries.

\i What happens to a wave pulse as it propagates
through a dispersive material?

\i Glass is a dispersive material for light with
the wave speed for red light being greater than
the wave speed for violet light.
Which color (red or violet) will be bent more 
(i.e., refracted through a bigger angle) as it passes from air 
into glass.

\i Explain why sound diffracts through a doorway but 
visible light does not?

\i (a) What must the wavelength of an electromagnetic
wave be in order for it to diffract through a doorway
like sound?
(b) What frequency does this correspond to?
(Recall that $v=3\times 10^8~{\rm m/s}$ for the speed
of light in air.)

\i A stationary source of sound has a (fundamental) 
frequency $f=440~{\rm Hz}$.
What is the observed frequency if an observer moves
(a) away from the source at a speed $v_o$ equal to 
the speed of sound in air $v=346~{\rm m/s}$, and
(b) toward the source at a speed equal to the speed of sound?

\een

