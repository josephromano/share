\section{Homework 2}

\noindent
Show all work.

\ben
\i Draw a graph of displacement versus time for 
general periodic motion.
Show three complete cycles of the motion.
Label the axes, and clearly indicate the period
$T$ and amplitude $A$ of the motion.

\i Draw a graph of displacement versus time for 
simple harmonic motion.
Show three complete cycles of the motion.
Label the axes, and clearly indicate the period
$T$ and amplitude $A$ of the motion.

\i What type of force produces simple harmonic motion.

\i Using the relation $2\pi~{\rm rad} = 360^\circ$,
determine the number of radians in an angle of
(a) $90^\circ$, (b) $270^\circ$, (c) $30^\circ$, and
(d) $60^\circ$.

\i A $1~{\rm kg}$ mass attached to a spring oscillates
up and down with a period of $T=2~{\rm s}$.
(a) What is the corresponding frequency of oscillation?
(b) Is this a frequency that we can hear with our ears?

\i Suppose we replace the $1~{\rm kg}$ mass from the 
previous problem with a $2~{\rm kg}$ mass.
What is the new period of oscillation?

\i A child and an adult go to a park and swing on 
neighboring swings, which have the same length.
Suppose the adult weighs twice as much as the child.
Does the child swing back-and-forth faster than the 
adult?
If not, explain why.

\i Give two examples of a damped oscillation.

\i In order to drive a damped oscillation so that 
the amplitude of the resulting motion is as large
as possible, how should the driving frequency
$f$ and natural frequency of the oscillation $f_0$
be related to one another?

\i Given two examples of resonance.

\een

