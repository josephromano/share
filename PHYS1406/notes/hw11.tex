\section{Homework 11}

\noindent
Show all work.

\ben

\i List the main parts of a piano.

\i Why doesn't a harpsichord have a large dynamic range?

\i Why do pianos have cast iron frames?

\i What is the purpose of having wrapped strings instead of
solid strings with the same linear mass density?

\i (a) How many piano strings are there for most keys?
(b) Are the strings for the same key tuned to the same frequency?

\i Draw a picture showing the attack and decay transient
of a piano note.

\i (a) Are the vibrational frequencies of a piano string
equal to harmonics of the fundamental frequency?
(b) If not, are the corresponding vibrational frequencies 
larger or smaller than those of pure harmonics?

\i What is the frequency difference between an equal-tempered
fifth and a perfect fifth?

\i (a) Explain what is meant by stretched tuning.
(b) Are all of the notes tuned to equal temperament?

\i Draw a Railsback curve, labeling the horizontal and
vertical axes.

\een

