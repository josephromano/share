\section{Matlab computer demonstrations}

The following is a list of matlab routines that can be used to 
illustrate a particular music or physics concept.

\bi

\i addsines.m: 
add two sine waves with the same amplitude, frequency, 
but with a phase offset $\phi$ (in degrees)

\i aliasing.m: 
script to illustrate aliasing for a sine wave sampled too slowly

\i amplitudemodulation.m:
combine carrier and signal waves using amplitude modulation

\i auralharmonics.m:
illustrate non-linear response of the human ear

\i bandpass.m:
apply bandpass filter 
(routine by fjsimons-at-alum.mit.edu)

\i bar\_free\_notension.m:
find discrete frequencies for free bar (no tension)

\i beats.m:
add two sine waves with the same amplitude and phase,
but different frequencies

\i bowedstring.m:
illustrate the physics of a bowed violin string

\i comparetemperaments.m:
compare equal, pythagorean, mean-tone and just temperaments
 
\i dampeddrivengui.m:
integrate $F = ma$ for damped, driven harmonic motion

\i doublependulum.m:
small oscillations of coplanar double pendulum

\i fifths.m:
compare two different fifths (C--G) and (C\#--Ab) in different temperaments

\i filter\_recorder.m:
record sound, filter, and calculate spectrogram

\i forcedoscillator.m:
illustrate chaotic motion of driven oscillator

\i fourieranalyze.m:
fourier analyze a time series

\i fourierdecompose.m:
calculate fourier coefficients for plucked guitar string fixed at both ends

\i fouriersynthesize.m:
fourier synthesizer using amplitude and phase information from first $N$ harmonics

\i fouriersynthesizeScript.m:
script for fouriersynthesize

\i fouriersynthesizegui.m:
fourier synthesizer using first eight harmonics

\i fouriersynthesizesound.m:
fourier synthesize sound using amplitude and phase information from first $N$ harmonics

\i fouriersynthesizesoundScript.m:
script for fouriersynthesizesound

\i frequencymodulation.m:
combine carrier and signal waves using frequency modulation

\i harmonics.m:
calculate nearest equal-tempered frequencies for first 8 harmonics of $f_0$

\i hipass.m:
apply high-pass filter
(routine by fjsimons-at-alum.mit.edu)

\i inharmonicity.m:
calculates the deviation of the nth partial from the exact nth harmonic for a 
real piano string

\i intervals.m:
calculate frequency ratios for musical intervals and compare to equal temperament

\i just.m:
calculate frequency ratios in just temperament and compare to equal temperament

\i loudness.m:
plot loudness versus intensity curves

\i lowpass.m:
apply low-pass filter
(routine by fjsimons-at-alum.mit.edu)

\i meantone.m:
calculate frequency ratios in mean-tone temperament and compare to equal temperament

\i multiplesources.m:
plot various quantities versus number of sources

\i note2freq.m:
convert note to frequency

\i periodicmotion.m:
numerically integrate 1-D equation for periodic motion 
where $F(x) = -{\rm sign}(x) k |x|^p$

\i phasemodulation.m:
combine carrier and signal waves using phase modulation

\i playchord.m:
play three notes in unison

\i playchromaticscale.m:
play chromatic scale

\i playdiatonicscale.m:
play diatonic scale in major interval order

\i playinterval.m:
play two notes in unison (notes specified by name)

\i playintervalFreq.m:
play two notes in unison (notes specified by frequency)

\i playnote.m:
play note (note specified by name)

\i playnoteFreq.m:
play note (note specified by frequency)

\i playrecordedsound.m:
ply recorded sound (if reverse=1, reverse sound in time domain)

\i playshepardtone.m:
play notes an octave apart in equal temperament
with intensities weighted by a gaussian centered at C4

\i playsound.m:
play sound (if reverse=1, reverse sound in time domain)

\i playrecordedsound.m:
play recorded sound that was saved to a file 
(if reverse=1, reverse sound in time domain)

\i pluckedstring.m:
illustrate physics of a plucked string

\i pluckedstringharmonics.m:
determine fourier coefficients for a plucked string

\i pythagorean.m:
calculate frequency ratios in pythagorean temperament and 
compare to equal temperament

\i ratio2cents.m:
convert ratio of frequencies to an interval in cents

\i recordsound.m:
record sound and fourier analyse
(if reverse=1, reverse sound in time-domain)

\i recordsoundandsave.m:
record sound and write to .mat file

\i shepardintensity.m:
plot gaussian intensity distribution for shepard tones

\i shepardscale.m:
play shepard tone scale

\i sound\_spectrogram.m:
record sound and calculate spectrogram

\i standingwaves.m:
demonstrate standing waves on a string fixed at
both ends (variable freq)

\i sumsines.m:
gui for adding two sine waves with variable amplitude, frequency,
and phase

\i tape\_recorder.m:
record sound and fourier analyse 
(allow forward/reverse and speed adjustment)

\i vibratingstring.m:
small transverse vibrations of a stretched string fixed at both ends
(approximate string as N discrete mass points)

\i vibratingstringScript.m:
script for running vibratingstring

\ei

