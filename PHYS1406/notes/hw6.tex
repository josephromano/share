\section{Homework 6}

\noindent
Show all work.

\ben

\i The effective area of an ear drum is 
approximately $5.5\times 10^{-5}~{\rm m}^2$.
The pressure variation for ordinary 
conversation is approximately $10^{-2}~{\rm N/m}^2$.
Calculate the force on an eardrum during ordinary
conversation.

\i Calculate the logarithms of the 
following numbers: 2, 10, 1000, 1/2, 0.01.

\i In what sense is the basilar membrane
like a piano keyboard?

\i What is the sound intensity level 
at 100~Hz of a pure tone having a sound loudness
level of 50~phon?

\i What weighting (A or C) should you use 
for a sound level meter if you want it to
measure the sound loudness level in phon?

\i What is the relationship between dB,
phon, and sone?

\i Two violins playing simultaneously 
each produce sound waves with
amplitude $p=10^{-2}~{\rm N/m}^2$.
Calculate the amplitude $p_{\rm tot}$ 
of the resultant wave assuming that the waves add:
(a) in phase with one another, and
(b) incoherently (which is the normal case).

\i Suppose one violin has a sound intensity 
level $L_p=70~{\rm dB}$.
Calculate the sound intensity level for:
(a) two violins playing simultaneously,
(b) ten violins playing simulatenously.

\i Suppose one violin has a subjective loudness
level $S=8~{\rm sone}$.
Calculate the subjective loudness level for:
(a) two violins playing simultaneously,
(b) ten violins playing simulatenously.

\i According to OSHA, 
(a) what is the maximum
duration per day for noise exposure to a 
piano being played fortissimo?
(b) What is the maximum duration per day
if you listen to two pianos simultaneously, each being 
played fortissimo?

\een

