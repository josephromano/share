\section{Tentative syllabus}

\begin{verbatim}
Elementary Physics of Sound and Music I and II (PHYS 1305, 1307)

Course description:
-------------------
A two-course sequence introducing students to the physics of
sound.  Topics to be covered over the two semesters include:
(i) physics of vibrations and waves; (ii) perception of sound; 
(iii) scales and tuning systems; (iv) instruments and voice;
(v) architectural acoustics; (vi) recording and reproduction 
of sound; and (vii) electronic musical instruments.  

The two-course sequence will be required of all music majors.
Non-music majors wanting to fulfill a general education core
requirement in science are also welcome to take these courses.  

Both courses will be taught in an integrated lecture-lab format
with many hands-on demonstrations.  Although the required 
mathematics will be developed as needed, a working knowledge 
of college algebra and basic trigonometry will be helpful.

Tentative breakdown of topics:
------------------------------
SEMESTER I
I. Physics of vibrations and waves

- vibrations and oscillations 
- simple harmonic motion
- damped and driven oscillations
- fundamental properties of waves
- superposition
- beats 
- resonance and standing waves
- doppler effect
- fourier analysis

II. Perception of sound

- hearing and the human ear
- pitch
- timbre (harmonic content of a sound)
- loudness and sound intensity scales
- logarithms and decibels
- OSHA regulations regarding sound intensity
- psychology of music

III. Scales and tuning systems

- historical development of pitch
- pentatonic scale
- pythagorean, just, and equal temperament systems
- piano tuning 

IV. Instruments and voice

- stringed instruments (plucked and bowed)
- piano 
- woodwinds and pipe organs
- brass
- percussion
- effects of temperature and humidity on instruments
- human voice and singing

--------------------
SEMESTER II
V. Architectural acoustics

- absorption, reflection, reverberation, and modeling
- sound reinforcement
- criteria in acoustical design
- problems in acoustical design
- design of rehearsal rooms and auditoriums
- concert halls

VI. Recording and reproduction of sound

- electrical circuits and Ohm's law
- microphone, amplifier, and loud speakers
- analog and digital reproduction of sound
- tape recorders
- electronic filters and effects
- CDs
- sampling, compression and mp3 format

VII. Electronic musical instruments

- electronic oscillators
- analog and digital synthesizers
- keyboards 
- MIDI standard

Demonstration equipment:
-----------------------
- Audio speakers ($500)
- Audio amplifier ($150)
- Equalizer ($100)
- Microphone(s) (Music dept.)
- oscilloscope ($400, borrow from physics)
- spectrum analyzer (borrow from physics)
- computer for data analysis and to connect to hardware above.
  (No need to buy one, borrow from physics)
- spectrum analyzer software (free, or $99)
  <http://www.sygyt.com/en/free-spectrum-analyzer>
- Matlab with signal processing toolbox (physics)
- tuning forks
- metronome
- several sound level meters (~$200 a piece)
- various musical instruments as needed:
- penny whistle, flute, guitar, violin, piano, ...
- audio recorders (smartphones)

Textbook:
---------
At the level of 
"The acoustical foundations of music" by Backus, 
"The physics of sound" by Berg and Stork, or 
"The science of sound" by Rossing.  

Additional reading material will be appropriate articles from
Scientific American or other such journals.

Grading:
--------
Combination of tests and group or individual projects (TBD)

\end{verbatim}
 
