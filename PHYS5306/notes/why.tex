\documentclass[10pt]{article}
\usepackage{amssymb,amsmath,amsthm,longtable}
\usepackage{latexsym}
\usepackage{placeins}
\usepackage{graphicx}
\usepackage{caption}
\usepackage{textcomp}
\captionsetup{width=5.75in}
\usepackage[top=1in, bottom=1in, left=1in, right=1in]{geometry}

\numberwithin{equation}{section}
\usepackage{amsfonts}
\usepackage{upgreek}
\usepackage{bm}
\usepackage{dsfont}
\usepackage{dcolumn}
\usepackage{epsfig}
\usepackage{graphics}
\usepackage{subfigure}

% begin equation, itemize, etc.

\def\be{\begin{equation}}
\def\ee{\end{equation}}
\def\bi{\begin{itemize}}
\def\ei{\end{itemize}}
\def\ben{\begin{enumerate}}
\def\een{\end{enumerate}}
\def\i{\item{}}
\newcommand{\bs}[1]{\boldsymbol{#1}}
\newcommand{\mb}[1]{\mathbf{#1}}
\renewcommand{\vec}[1]{\mathbf{#1}}
\def\d{{\rm{d}}}

\begin{document}
%%%%%%%%%%%%%%%%%%%%%%%%%%%%%%%%%%%%%%%%%%%%%%%%%%
% noindent, space between paragraphs
\setlength{\parindent}{0pt}
\setlength{\parskip}{\medskipamount}
%%%%%%%%%%%%%%%%%%%%%%%%%%%%%%%%%%%%%%%%%%%%%%%%%%

%%%%%%%%%%%%%%%%%%%%%%%%%%%%%%%%%%%%%%%%%%%%%%%%%%%%%%%%%%%%%%%%%%%%%%
\title{Why should I learn this stuff?}
\maketitle
%\tableofcontents

{\bf General remarks:}
Since very few of you will go on to careers that will 
require you to use classical mechanics as part of your
job, one is justified in asking why you should spend a 
whole semester learning graduate-level classical mechanics, 
when you could be learning other ``more practical" things.
One answer to this question (which applies as well to 
other core courses in physics) is that the problem-solving 
skills and mathematical techniques that you will learn 
in this course are applicable to {\em other} areas 
of physics and science.
Having extra ``tools in your toolbox" is often very helpful
when it comes to tackling unsolved research problems.
One never knows when a certain technique or approach will 
lead to a solution.

Another answer is that, although the number may be small,
some of you {\em will} actually 
end up working in fields where certain aspects of classical 
mechanics will be required or at least be very helpful.
For example, a theoretical physicist studying studying 
classical or quantum field theory (particle physics, 
relativity, etc.)\ will often write 
down a Lagrangian as the first step in defining or 
solving a problem. 
Knowing the associated classical Hamiltonian and 
Poisson bracket structure then allows one to transition 
to quantum theory with its Hamiltonian operator and commutators.

Many mechanical, civil, and aerospace engineers 
will also use classical mechanics in their work, as an analysis 
of the forces and torques acting on a system determines 
whether or not a bridge will collapse or a rocket will leave 
the launch pad, etc.
Astrophysicists make use of concepts from central force 
motion when studying the gravitational interaction between 
binary star systems, galaxy rotation curves, etc.

And, finally, don't forget the numerous applications of
classical mechanics in your everyday lifes.  
An apple falling from a tree, 
the orbit of the Moon, a spinning ice skater,
%speeding up as she pulls her arms closer to her body, 
the trajectory of football pass, the impact of 
a tennis racket with a tennis ball, $\ldots$, are all described 
by the laws of classical mechanics.
Having a solid understanding of classical mechanics will 
therefore give you a better understanding and appreciation 
of much of the world around you.

%%%%%%%%%%%%%%%%%%%%%%%%%%%%%%%%%%%%%%%%%%%%%%%%%%%%%%%%%%%%%%%%%%%%%%
\section{Lagrangian mechanics (\S1-5)}

Lagrangian mechanics is an alternative formulation of
Newton's laws applied to a system of interacting particles.
It is formulated in terms of scalar quantities (kinetic
and potential energies) as opposed to vectors
(accelerations, forces, torques), and as such is mathematically
easier to work with.
Constraint (reaction) forces can be ignored in the 
Lagrangian formalism by working with generalized coordinates
that specify the true degrees of freedom of the system.
If desired, constraint forces can be included in the 
Lagrangian formalism by using the method of Lagrange multipliers.

Conservation of energy, momentum, and angular momentum are 
simply described in the Lagrangian formalism, being related 
to symmetries of the Lagrangian with respect to time 
translations, space translations, and rotations, respectively.

%%%%%%%%%%%%%%%%%%%%%%%%%%%%%%%%%%%%%%%%%%%%%%%%%%%%%%
\section{Non-inertial reference frames (\S39)}

Newton's law $\vec F=m \vec a$ and the Lagrangian
$L=T-U$ are valid only in an inertial frame of 
reference.
But most reference frames, like one attached to
the surface of the Earth, are non-inertial 
(due to translational acceleration or rotational motion).
In a non-inertial frame, Newton's law acquires additional
{\em fictitious} force terms associated with the 
acceleration of the reference frame.
We must include these fictitious force terms (e.g.,
the Coriolis force) to properly describe projectile motion
on the surface of the Earth.
Firing artillery shells on a battle field without taking
into account the deflection caused by the Coriolis force
would lead you to miss the target.
In addition, the precession of the plane of oscillation of 
a pendulum on the surface of the Earth is a direct demonstration 
of Earth's rotational motion (Foucault's pendulum).

%%%%%%%%%%%%%%%%%%%%%%%%%%%%%%%%%%%%%%%%%%%%%%%%%%%%%%
\section{Hamiltonian mechanics (\S40)}

Hamiltonian mechanics is a reformulation of 
Lagrangian mechanics which takes the generalized
coordinates $q_i$ and generalized momenta 
$p_i\equiv \partial L/\partial q_i$
% $i=1,2,\cdots, s$ 
as the fundamental variables.
Hamilton's equations are first-order differential
equations for both the $p$'s and $q$'s, which are 
often simpler to solve than the corresponding 
second-order Lagrange's equations for the $q$'s.

The true value of the Hamiltonian formulation arises 
when one considers transformations that mixes the 
$q$'s and the $p$'s.  
Such transformations, which preserve Hamilton's equations,
are called {\em canonical transformations},
and by judicious choice of these transformations, the 
system of equations is easier to solve.
In the Hamiltonian formulation, conserved quantities
turn out to be generators of symmetry transformations of
the Hamiltonian.
Unfortunately, we will not have time this semester to 
cover these latter items.
They are discussed in Chapter VII, \S 42, 45, 47.

%%%%%%%%%%%%%%%%%%%%%%%%%%%%%%%%%%%%%%%%%%%%%
\section{Conservation laws (\S6-10)}

Conserved quantities are useful since they don't change
during the motion of a system of particles.
They reduce the number of equations of motion that we
need to integrate.
For example, using conservation of energy and angular
momentum for central force problems allows to us to 
solve {\em first-order} differential equations for $r$ and $\phi$, 
instead of more complicated second-order equations.

The motion of the center of mass (COM) of a system of 
particles can be thought of as the motion of the system 
{\em as as whole}.  
For a closed system, the COM moves with constant velocity, 
which can be set to zero by working in the COM frame.
This simplifies the description of the motion of the 
individual particles.

%%%%%%%%%%%%%%%%%%%%%%%%%%%%%%%%%%%%%%%%%%%%%
\section{Central force motion (\S11, 13-15)}

Central force motion arises whenever you have 
two bodies that interact via a potential that depends
only on the distance between the two bodies.
It describes, for example, the motion of the planets 
around the Sun, the orbits of communication satellites 
around the Earth, etc.
Kepler's laws of planetary motion are all derivable
using the formalism of central force motion for the 
potential $U(r)= -\alpha/r$, with $\alpha>0$.

%%%%%%%%%%%%%%%%%%%%%%%%%%%%%%%%%%%%%%%%%%%%%
\section{Collisions and scattering (\S16-20)}

Using only conservation of momentum and energy 
for elastic collisions or disintegrations of 
particles, we can determine a lot about the 
end state of a collision without knowing the 
explicit form of the interaction that causes it.
Additionally knowing the scattering potential 
allows us to fully determine the 
scattering angles and differential cross section 
for the incident and target particles, as a function 
of the incident energy and impact parameter of 
the incoming particle.

Much of experimental high-energy physics involves the 
scattering of high-energy subatomic particles (e.g., 
think of the Large Hadron Collider).
Gravitational ``sling-shot" trajectories, which are 
used to get satellites to distant planets without the use of 
additional fuel, are a combination of inverse-square-law-force
bound orbits and scattering of one mass off of another.
 
%%%%%%%%%%%%%%%%%%%%%%%%%%%%%%%%%%%%%%%%%%%%%
\section{Small oscillations (\S21-23)}

Simple harmonic motion (like that for a mass attached to 
a spring, $F=-kx$) occurs whenever one has small oscillations 
about a position of stable equilibrium.
This occurs in many different circumstances, as it 
only requires that the system of interacting 
particles be slightly perturbed away from a stable 
equilibrium configuration.
Mathematically, such oscillations occur because
the potential is well-approximated by a quadratic 
function in the neighborhood of a position of stable 
equilibrium.

%%%%%%%%%%%%%%%%%%%%%%%%%%%%%%%%%%%%%%%%%%%%%
\section{Rigid body motion (\S31-36, 38)}

To describe the complicated translational and rotational 
motion of a football (or any extended object) as 
it moves through space, we need to go beyond the 
particle approximation, which was used in the previous sections.
The angular velocity, angular momentum, and any external
torques that may act on the body are the key quantities that 
enter such an analysis.
Spinning tops, gyroscopes, and the precession of the 
Earth's axis are examples that fall within the realm of 
rigid-body motion.

\end{document}
